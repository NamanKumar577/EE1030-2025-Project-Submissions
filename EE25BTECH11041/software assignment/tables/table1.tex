\begin{table}[h!]
\centering
\renewcommand{\arraystretch}{1.3}
\begin{tabular}{|p{3cm}|p{5cm}|p{7cm}|}
\hline
\textbf{Algorithm} & \textbf{Advantages} & \textbf{Disadvantages} \\ 
\hline

\textbf{QR Algorithm with Householder + Wilkinson Shift} 
& 
\begin{itemize}
    \item Very fast and reliable for symmetric or Hermitian matrices.
    \item Wilkinson shift accelerates convergence significantly.
    \item Householder reflections ensure numerical stability.
\end{itemize}
& 
\begin{itemize}
    \item Complex to code manually (requires multiple matrix factorizations).
    \item Computationally expensive for large matrices due to repeated QR decompositions.
\end{itemize} 
\\ \hline

\textbf{Divide-and-Conquer Algorithm}
& 
\begin{itemize}
    \item Highly efficient for large symmetric matrices.
    \item Well-suited for parallel computation and modern libraries.
\end{itemize}
& 
\begin{itemize}
    \item Complicated to implement from scratch.
    \item Less stable for closely spaced or small eigenvalues.
    \item Requires tridiagonal form preprocessing.
\end{itemize} 
\\ \hline

\textbf{Jacobi Algorithm (One-Sided for SVD)}
& 
\begin{itemize}
    \item Easiest to implement in C — requires only basic matrix operations.
    \item High numerical stability due to use of orthogonal rotations.
    \item Intuitive conceptually (pairwise annihilation of off-diagonal terms).
    \item Produces accurate eigenvalues and eigenvectors without forming $A^TA$ explicitly.
\end{itemize}
& 
\begin{itemize}
    \item Converges slowly for large or ill-conditioned matrices.
    \item Not the fastest for very large datasets compared to QR-based methods.
\end{itemize} 
\\ \hline

\textbf{Bisection Method}
& 
\begin{itemize}
    \item Reliable for finding specific eigenvalues within an interval.
    \item Efficient for tridiagonal matrices.
\end{itemize}
& 
\begin{itemize}
    \item Does not compute eigenvectors.
    \item Slow convergence without accurate interval estimation.
    \item Limited to symmetric matrices.
\end{itemize} 
\\ \hline

\textbf{Standard QR Algorithm (without Shift)}
& 
\begin{itemize}
    \item Simpler than the shifted version.
    \item Works for general (non-symmetric) matrices.
\end{itemize}
& 
\begin{itemize}
    \item Much slower convergence.
    \item Poor efficiency for large-scale problems.
\end{itemize} 
\\ \hline

\end{tabular}
\caption{Comparison of eigenvalue and SVD algorithms with emphasis on implementation difficulty and stability. The Jacobi method, though slower, is preferred for its simplicity and robustness in C-based image compression applications.}
\label{tab:svd_algorithms}
\end{table}
